% abstract.tex (Abstract)

\addcontentsline{toc}{chapter}{Abstract}

\chapter*{Abstract}
The concept of Internet of Things (IoT) offers immense benefits by
enabling devices to leverage networked resources thereby making more intelligent
decisions. The numerous heterogeneous connected devices that exist throughout
the IoT system creates new security and privacy concerns. Some of these concerns can
be overcome through data trust, transparency, and integrity, which can be
achieved with data provenance. Data provenance also known as data lineage provides a history of transactions that occurs on a data object from the time it was created to its current state.Data provenance has immense benefits in detecting and mitigating current and future vulnerability attacks and has applications in anomaly detection, access control, and digital forensics.  \par This dissertation looks at provenance with a focus on IoT devices. It takes a holistic approach in the creation, security and applications of provenance data. We create a provenance aware system for IoT devices. This ensure trust and helps establish causality for decisions and actions taken by an IoT connected system. As a result of the amounts of data that is generated from our data provenance system, there arises an issue of running out of memory.To address this issue,we look at prior work done in the area of data compression and graph summarization. We address this by proposing a novel data pruning technique.We conclude by looking at the applications of provenance for security of malicious attacks and anomaly detection of malicious threats.

