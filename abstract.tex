% abstract.tex (Abstract)

\addcontentsline{toc}{chapter}{Abstract}

\chapter*{Abstract}
The concept of Internet of Things (IoT) offers immense benefits by
enabling devices to leverage networked resources thereby making more intelligent
decisions. The numerous heterogeneous connected devices that exist throughout
the IoT system creates new security and privacy concerns. Some of these concerns can
be overcome through data trust, transparency, and integrity, which can be
achieved with data provenance. Data provenance, also known as data lineage, provides a history of transactions that occurs on a data object from the time it was created to its current state. Data provenance has been explored in the areas of scientific computing and business to track the workflow of processes, as well as in field of computer security for forensic analysis and intrusion detection. Data provenance has immense benefits in detecting and mitigating current and future malicious attacks.  \par In this dissertation proposal, we explore the integration of provenance within the IoT architecture. We take an in-depth look at the creation, security and applications of provenance data in mitigating malicious attacks. We define our model constructs for representing IoT provenance data. We propose a provenance aware system that collects provenance data in an IoT framework. We focus on a holistic approach in which provenance information is represented at all layers of the IoT architecture. This system ensures trust and helps establish causality for decisions and actions taken by an IoT connected system. However, there are existing issues with devices running memory space, and this is presented as a consequence of the amount of data generated by our data provenance collection system, as well as resource constrains on low memory embedded systems. To address this issues, we propose a framework that provides an efficient storage of provenance data contained in an IoT architecture. We evaluate the effectiveness of our proposed frameworks by looking at an application of provenance data for the security of malicious attacks. We focus on evaluation using an Intrusion Detection System, which detects malicious threats that exists in an IoT framework.


