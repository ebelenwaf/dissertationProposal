\chapter{Background Information}

This chapter describes key concepts of data provenance, IoT characteristics, and provenance models. It outlines differences that exists between provenance and log also provenance and metadata. 

\section{Data Provenance}
The oxford English dictionary defines provenance as the place of origin or earliest known history of something.  An example of provenance can be seen with a college transcript. A transcript can be defined as the provenance of a college degree because it outlines all of the courses satisfied in order to attain the degree.
\par In the field of computing, data provenance also known as  data lineage can be defined as the history of all transformations performed on a data object from the its creation to its current state. Cheney et al \cite{cheney_provenance_2009} decribes provenance as the origin and history of data from its lifecycle. Buneman et al \cite{buneman_why_2001} describes provenance from a database perspective as the origin of data and the steps in which it is derived in the database system.  We formally define provenance as follows:


\begin{definition}

Data provenance of an object can be defined as a comprehensive history of transformations that occurs on that object from its creation to its present state.

\end{definition}

Provenance involves two granularity levels: fine-grained provenance and coarse-grained provenance. Fine-grained provenance \cite{glavic_case_2011} entails the collection of a detailed level of provenance data. Coarse grained provenance, on the other hand, is provenance collection of a summarized version of provenance data. Data provenance has immense applications and has been explored in the areas of scientific computing \cite{groth, altintas} to track how an experiments are produced, in business to determine the work-flow of  a process, in field of computer security for forensic analysis and intrusion detection \cite{bates_towards_2013, muniswamy_reddy_provenance_2010, muniswamy_reddy} . Provenance denotes the who, where and why of data \cite{cheney_provenance_2009}. An example of provenance for a software system is a web server's log file. This file contains metadata for various request and response time and the ip address of all host systems that requests information from the server. Provenance data is represented as an acyclic graph which denotes casual relationship and dependencies between entities. Provenance ensures trust and integrity of data. The information in which provenance offers can be used in digital forensics to investigate the cause of a malicious attack and also in intrusion detection systems to further enhance the security of computing devices. 





\subsection{Provenance Characteristics}

Since provenance denotes the who, where and why of data transformation, it is imperative that we collect sensor and actuator reading that satisfies the required conditions. The characteristics of data provenance are outlined in detail below.


\begin{itemize}

\item Who: This characteristic provides information on who made modifications to a data object. Knowing the ``who" characteristic is essential because it maps the identity of modification to a particular data object. An example of who in an IoT use case is a sensor device ID.

\item Where: This characteristic denotes location information in which data transformation was made. This provenance characteristic could be optional since not every data modification contains location details.

\item When: This characteristic denotes the time information in which data transformation occurred. This is an essential provenance characteristic. Being able to tell the time of a data transformation allows for varied use scenarios of tracing data irregularities.

\item How: This characteristic denotes the process in which transformation is applied on a data object. A use case for IoT can be seen in the various operations(create, read, update, and delete) that could be performed on a file object.

\end{itemize}


 
\section{Internet of Things(IoT)}
There is no standard definition for IoT, however, researchers have tried to define the concept of connected ``things". The concept of IoT was proposed by Mark Weiser in the early 1990 \cite{Mattern} which represents a way in which the physical objects, ``things", can be connected to the digital world. Gubbi et al \cite{park_provenance-based_2012} defines the Internet of Things as  an interconnection of sensing and actuating devices that allows data sharing across platforms through a centralized framework. We define the internet of things(IoT) as follows:

\begin{definition}
The Internet of Things (IoT) can be defined as a network of heterogeneous devices with sensing and actuating capabilities communicating with each other thereby enabling each devices become smarter through data analytics and computation. 

\end{definition}

The notion of internet of things has been attributed to smart devices. It is believed that the interconnectivity between various heterogeneous devices allows for devices to share information in a unique manner. Analytics is the driving force for IoT. With analytics, we are able to learn user patterns  and this information in turn can be used to make smarter device decisions. This notion of smart devices can be seen in various commercial applications such as smartwatches to smart thermostats that automatically learns a user patterns. The ubiquitous nature of these devices make them ideal choices to be included in consumer products. 

\par With the recent data explosion due to the large influx in amounts of interconnected devices, information is disseminated at vast rate and with this increase involves security and privacy concerns. Creating a provenance-aware system is beneficial to IoT because it ensures the trust and  integrity of interconnected devices. Enabling provenance collection in IoT device allows devices to capture valuable information which enables backtracking in an event of a malicious attack. We take a holistic approach to provenance collection by looking at how provenance information is collected across an IoT architectural framework.

 
  \subsection{IoT Architecture}

IoT architecture represents a hierarchical view of how information is disseminated across multiple layers contained in an IoT framework; from devices which contains sensing and actuating capabilities to massive data centers(cloud storage). Knowing how information is transmitted across various layers allows a better understanding on how to model the flow of information across various actors contained in an IoT hierarchy. The IoT architecture consists of four distinct layers: The sensor and actuator layer, device layer, gateway layer and the cloud layer. The base of the architectural stack consist of sensors and actuators which gathers provenance information and interacts with the device layer. The device layer consists of devices(e.g mobile phones, laptops, smart devices) which are responsible for aggregating data collected from sensors and actuators. These device in turn forwards the aggregated data to the gateway layer. The gateway layer is concerned with the routing and forwarding of data collected from the device later. It could also serve as a medium of temporal storage and data processing. The cloud layer is involved with the storage and processing of data collected from the gateway layer. Figure 2 displays the IoT architecture and the interactions between the respective  layers. It is important to note that the resource constraints decreases as you go up the architectural stack with the cloud layer having the most resources(memory, computation) and the sensor layer having the least amount of resource capability. The arrows illustrates the interaction between data at various layers on the architecture.


\begin{figure}[h]
\begin{center}

\includegraphics[height=3.0in]{iot_architecture.png}
\end{center}
\caption{IoT Architecture Diagram}

\end{figure}



\section{Comparison of Provenance with Log data and metadata}

Provenance data, log data and metadata are key data concepts that often are used interchangeably. We try to address the difference and similarities that might exist between provenance data, metadata and log data.

\subsection{Provenance and Metadata}
Metadata and provenance are often considered related but yet subtle exists. Metadata contains descriptive information about data. Metadata can be considered as provenance when there exists relationship between objects and they explain the transformation that occurs. For example, a web server can have metadata information such as the host and the destination IP address which might not be considered as provenance information for a specific application. It could also contain timestamps with location of IP addresses which could be considered provenance information. In summary,  metadata and provenance are not the same, however an overlap exists. Metadata contains valuable  provenance information but not all provenance information is metadata. 


\subsection{Provenance and Log data}
Log data contains information about the activity of an object in an operating system or process. Log data can be used as provenance. It contains data trace specific to an application domain. Log files might contain unrelated information such as error messages, warnings which might not be considered as provenance data. Provenance allows for specified collection of information that relates to the change the internal state of an object. In summary, log data could provide insight to what provenance data to collect. 

\section{Model for representing provenance for IoT}

In order to ensure that we properly represent the right kind of provenance information, we need to satisfy the who, where, how, and what of data transformations. Provenance data is represented using a provenance model, PROV\-DM which is represented in serialized form as JSON object. This model displays the causal relationship of data objects contained in an IoT architectural stack. In this model is contained information such as sensor readings, device name, and device information. This information is mapped into an appropriate provenance data model to allow for interoperability and visualization. There are two models for representing provenance. These models \cite{prov_dm} have been applied in various literature and is considered a standard for representing provenance. Details on the provenance models are outlined below:

\subsection{Open Provenance Model(OPM)}

Open Provenance Model \cite{moreau_open_2011} is a specification that was derived as a result of a meeting at the International Provenance and Annotation Workshop (IPAW) in May 2006. OPM was created to address the need of allowing a centralized model for representing provenance data amongst various applications. It allows for interoperability between various provenance applications. The goal of OPM is to represent causality and dependency and also to develop a digital representation of provenance data regardless of if it is produced by a computer system. OPM is represented as a directed acyclic graph which denotes causal dependency between objects. The edges in the graph denotes dependencies with its source denoting effect and its destination denoting cause. The definitions of the edges and their relationships are denoted below: 

\begin{itemize}
\item wasGeneratedBy: This edge denotes a  relationship in which an entity(e,g artifact) is utilized by one or  more entities(e.g process). An entity can use multiple entities so it is important to define the role.  
\item wasControlledBy: This edge outlines a relationship in which an entity caused the creation of another entity.
\item used(Role): This edge denotes that an entity requires the services of another entity in order to execute.
\item wasTriggeredBy: This edge denotes a relationship represents a process that was triggered by another process
\item wasDerrivedFrom: This edge denotes a relationship which indicates that the source needs to have been generated before the destination is generated.
\end{itemize}

 Objects represents the nodes of the relationship graph. There are three object contained in the OPM model: artifact, process, agent. 

\begin{itemize}
\item
artifact: An artifact represents the state of an entity. An artifact is graphically represented by a circle.

\item
Process: A process is an event which takes place at a particular time due to an agent or an artifact. A process is represented by a square object.

\item 
Agent: Agents represents actors that facilitate the execution of a process. An agent is represented by a hexagon in an OPM graph.
\end{itemize}

OPM is used to represent all previous and current actions that have been performed on an entity and  the relationship between each entities contained in the graph. Figure 2 represents an example of an OPM acyclic graph with all of its causal dependencies.  

\begin{figure}[h]
\begin{center}
\includegraphics[height=3.0in]{opm_convention.PNG}
\end{center}
\caption{Edges and entities represented in OPM}
\label{autom}
\end{figure}

%\begin {center}
%\begin {tikzpicture}[-latex ,auto ,node distance =4 cm and 5cm ,on grid ,
%semithick ,
%state/.style ={ circle ,top color =white , bottom color = processblue!20 ,
%draw,processblue , text=blue , minimum width =1 cm}]
%\node[state] (C)
%{$1$};
%\node[state] (A) [above left=of C] {$0$};
%\node[state] (B) [above right =of C] {$2$};
%\path (A) edge [loop left] node[left] {$1/4$} (A);
%\path (C) edge [bend left =25] node[below =0.15 cm] {$1/2$} (A);
%\path (A) edge [bend right = -15] node[below =0.15 cm] {$1/2$} (C);
%\path (A) edge [bend left =25] node[above] {$1/4$} (B);
%\path (B) edge [bend left =15] node[below =0.15 cm] {$1/2$} (A);
%\path (C) edge [bend left =15] node[below =0.15 cm] {$1/2$} (B);
%\path (B) edge [bend right = -25] node[below =0.15 cm] {$1/2$} (C);
%\end{tikzpicture}
%\end{center}


\subsection{Provenance Data Model(Prov-DM)}

Another model for representing provenance data is the Provenance Data Model(PROV\-DM). PROV-DM is a W3C standardized extension of OPM. Prov-DM is a model that is used to depict causal relationship between entities, activities and , and agents(digital or physical).  It creates a common model that allows for interchange of provenance information between heterogeneous devices. It contains two major components: types and relations. 


\begin{itemize}

\item entity: An entity is a physical or digital object. An example of an entity is a file system, a process, or an motor vehicle.

\item Activity: An activity represents some form of action that occurs over a time frame.Actions are acted upon by an entity. An example of an activity is a process opening a file directory, Accessing a remote server.

\item Agent: An agent is a thing that takes ownership of an entity, or performs an activity. An example of an agent is a person, a software product, a process.
\end{itemize}

The figure below illustrates the various types contained in PROV\-DM and their representation. Entities, activities and agents are represented by oval, rectangle and hexagonal shapes respectively.

\begin{figure}[h]
\begin{center}

\includegraphics[width=4.0in]{prov_dm_1.PNG}
\end{center}
\caption{Prov-DM respresentation }
\end{figure}

Similar to the OPM, PROV\-DM does not keep track/estimate of future events. PROV relations are outlined below:


\begin{itemize}
\item wasGeneratedBy: This relation signifies the creation of an entity by an activity. 

\item used: This relation denotes that the functionality of an entity has been adopted by an activity.

\item wasInformedBy: This relation denotes a causality that follows the exchange of two activities.

\item wasDerievedFrom This relation represents a copy of information from an entity. 

\item wasAttributedTo: This denotes relational dependency to an agent. It is used to denote relationship between entity and agent when the activity that created the agent is unknown.

\item wasAssociatedWith:This relation denotes a direct association to an agent for an activity that occurs.This indicates that an agent plays a role in the creation or modification of the activity.

\item ActedOnBehalfOf: This relation denotes assigning authority to perform a particular responsibility to an agent. This could be by itself or to another agent.



\end{itemize}

Prov-DM contains similar yet subtle differences between OPM. Some of the difference between OPM and PROV\-DM are described below:

\begin{itemize}

\item The main components Artifact, Process and Agent in the OPM model are modified to Entity, Action, and Agent. 

\item Additional causal dependencies such as wasAttributedTo and actedOnBelafOf were included to represent direct and indirect causal dependencies respectively between agents and entities.

\end{itemize}

%Since PROV-DM is built ontop of OPM and contains easy to understand constructs of types and relations. We are envision are properly represented through the various types and relations, we choose this model to represent provenance data in our IoT architecture instead of OPM. 

\subsection{PROV-JSON}

PROV-JSON is a component of PROV-DM specification. It is used for representing PROV-DM data in JSON(Javascript Object Notation) format. It contains all of the components and relationships contained in PROV-DM. It also allows for easy serialization and deserialization of PROV-DM to JSON representation. JSON is a lightweight data format which is human readable and easy to parse. An example of PROV\-JSON format which includes the PROV\-DM types is illustrated below:

\begin{figure}[h]
\begin{center}

\includegraphics[height=5in]{prov_json.png}
\end{center}
\caption{PROV-JSON MODEL}

\end{figure}



\section{Overview of Relevant Compression techniques}

\subsection{Lossy Compression}

Lossy compression \cite{Sayood:2000:IDC:336428} is a form of data compression in which original data can be reconstructed with some tolerance on loosing some information contained in the original data. This enables higher compression ratio than other compression techniques. Lossy Compression is mostly used in applications that do not contain strict requirements as to loosing some information. An example of a lossless compression application can be seen in an image compression software where the quality of the image is not seen by the naked eye. It is also used in speech transmission. 

\subsection{Lossless Compression}
Lossless compression \cite{Sayood:2000:IDC:336428} is a data compression technique in which the original data is reconstructed without loss of any information. It can be used by applications which requires that the data is compressed and the original data be identical. An example of such an application that requires lossless compression is text compression.  Any compression that alters the structure of the original text results to a different meaning of the text. For instance, the sentence ``I have a black cat" would have a different meaning if any information is lost from it.



\subsubsection{Arithmetic Coding}

Arithmetic coding is a form of lossless compression which encodes a stream of characters into a variable size interval between [0,1). A probability is assigned to each characters contained in the string. A cumulative probability is used to calculate the interval of the respective character. Frequent characters are encoded with shorter codes than less frequent characters. 
